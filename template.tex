% Written by Daina Chiba (daina.chiba@gmail.com).
% It was mostly copied from two poster style files:
% beamerthemeI6pd2.sty written by
%	 	Philippe Dreuw <dreuw@cs.rwth-aachen.de> and 
% 		Thomas Deselaers <deselaers@cs.rwth-aachen.de>
% and beamerthemeconfposter.sty written by
%     Nathaniel Johnston (nathaniel@nathanieljohnston.com)
%		http://www.nathanieljohnston.com/2009/08/latex-poster-template/
% 
% Modified for UvA by
% Maarten de Rijke, April 2014
%
%---------------------------------------------------------------------------------------------------% 
% Preamble
% ---------------------------------------------------------------------------------------------------% 
\documentclass[final]{beamer}
\usepackage[orientation=landscape,size=a0,scale=1.2,debug]{beamerposter}
\mode<presentation>{\usetheme{UvAPoster}}
\usepackage[english]{babel}
\usepackage[latin1]{inputenc}
\usepackage[T1]{fontenc}
\usepackage{amsmath,amsthm, amssymb, latexsym}

\usepackage{array,booktabs,tabularx}
\newcolumntype{Z}{>{\centering\arraybackslash}X} % centered tabularx columns

% comment 
\newcommand{\comment}[1]{}

% (relative) path to the figures
\graphicspath{{figs/}}

\newlength{\columnheight}
\setlength{\columnheight}{105cm}
\newlength{\sepwid}
\newlength{\onecolwid}
\newlength{\twocolwid}
\newlength{\threecolwid}
\setlength{\sepwid}{0.024\paperwidth}
\setlength{\onecolwid}{0.24\paperwidth}
\setlength{\twocolwid}{0.4\paperwidth}
\setlength{\threecolwid}{0.19\paperwidth}

% ---------------------------------------------------------------------------------------------------% 
% Title, author, date, etc.
% ---------------------------------------------------------------------------------------------------% 
\title{\huge Are Neural Click Models Pointwise IPS Rankers?}
\author{Philipp Hager\inst{1} \and Maarten de Rijke\inst{1} \and Onno Zoeter\inst{2}}
\institute[shortinst]{\inst{1} University of Amsterdam \inst{2} Booking.com}
\date[Sep. 2022]{September, 2022}
%%% Put the name of conference here.
	\def\conference{CONSEQUENCES+REVEAL Workshop at RecSys '22} 
 %%% Put your e-mail address here.
 	\def\yourEmail{p.k.hager@uva.nl}


% ---------------------------------------------------------------------------------------------------% 
% Contents
% ---------------------------------------------------------------------------------------------------% 
\begin{document}
\begin{frame}[t]
	\begin{columns}[t]
    % -----------------------------------------------------------
    % Start the first column
    % -----------------------------------------------------------
    \begin{column}{\onecolwid}
      % -----------------------------------------------------------
      % 1-1 (first column's first block
      % -----------------------------------------------------------
\vskip3ex
      % -----------------------------------------------------------
      % 1-1
      % -----------------------------------------------------------
      \begin{alertblock}{Alerted Block}
        \baselineskip=.7\baselineskip
        \vskip1ex
        This is an \textbf{\alert{alerted}} block. \\
        Use alerted blocks to highlight important points.
        \vskip1ex
      \end{alertblock}

      \vskip3ex

      % -----------------------------------------------------------
      % 1-2
      % -----------------------------------------------------------
      \begin{block}{Normal Block}
        \baselineskip=.7\baselineskip

	This is a normal block. 

	\begin{itemize}
	\item Put
	\item your
	\item argument
	\item here
	\end{itemize}

	\end{block}

\vskip3ex

      % -----------------------------------------------------------
      % 1-3
      % -----------------------------------------------------------
      \begin{block}{Sample Block}
        \baselineskip=.7\baselineskip
\vskip2ex

	Put something here. You can put equations, too.

\begin{align}
f(x) &= \frac{1}{2\sqrt{\pi \sigma^2}}\exp \Big(- \frac{(x-\mu)^2}{2\sigma^2} \Big) \nonumber
\end{align}

\vskip2ex

	Normal distribution
	\begin{itemize}
	\item $x$ is a random variable.
	\item $\mu$ is the mean of $x$.
	\item $\sigma$ is the standard deviation of $x$
	\end{itemize}

      \end{block}


      \vskip3ex

      % -----------------------------------------------------------
      % 1-4
      % -----------------------------------------------------------
      \begin{block}{Sample Block}
        \baselineskip=.7\baselineskip
\vskip2ex
        You can put figures, too.

	\vskip2ex

	\begin{figure}
	\begin{center}
	\caption{Potential Endogeneity Problem}
		\includegraphics[width=9in]{figEndog}
	\end{center}
	\end{figure}
	
	\vskip2ex

	{\bf Put something here} 
	\begin{itemize}
	\item Put something here. Put something here.Put something here.
	\item Put something here. Put something here.Put something here.
	\item Put something here. Put something here.Put something here.
	\end{itemize}

      \end{block}

	\end{column}

    % -----------------------------------------------------------
    % Start the second column
    % -----------------------------------------------------------
    \begin{column}{\twocolwid}
    \begin{columns}[t,totalwidth=\twocolwid]
    	    \begin{column}{\threecolwid}
      % -----------------------------------------------------------
      % 2-1a
      % -----------------------------------------------------------
      \begin{block}{Sample Block (Left)}
        \baselineskip=.7\baselineskip

	Put whatever you want.
	\begin{itemize}
	\item Yes.
	\item This
	\item is
	\item just
	\item an
	\item example.
	\end{itemize}

	\vskip1ex
      \end{block}
	    \end{column}
	    \begin{column}{3.75ex}\end{column}
    	    \begin{column}{\threecolwid}

      % -----------------------------------------------------------
      % 2-1b
      % -----------------------------------------------------------
      \begin{block}{Sample Block (Right)}
        \baselineskip=.7\baselineskip

	Put whatever you want.
	\vspace{1ex}
	\begin{itemize}
	\item No.
	\item This
	\item is
	\item not
	\item a real
	\item poster.
	\end{itemize}
      \end{block}
	    \end{column}
	   \end{columns}
      \vskip2ex

      % -----------------------------------------------------------
      % 2-3
      % -----------------------------------------------------------
      \begin{alertblock}{Sample Block (Wide)}
	\begin{columns}[c] \column{.45\textwidth}
%      \vskip1ex
      {\bf This block is wide.}
      \begin{itemize}
      \item Put something here.Put something here.Put something here.
      \item Put something here.Put something here.
      \item Don't you get it? Put something here.
      \end{itemize}

%\vskip2ex

	% Phantom column
	\column{.01\textwidth}
	\begin{beamercolorbox}[wd=.1in,ht=2in]{cboxg}\end{beamercolorbox}
	\vskip2ex

	\column{.45\textwidth}
	\vskip2ex
      {\bf Put something here.}
      \vskip1ex
	\begin{tabular}{>{\color{myred}}rl}
	Something & = Blur blur blur\\
		& Yadi Yadi Ya \\ [1em]
	Anything & = Put something here. \\
		& OK?
	\end{tabular}

\end{columns}

	\vskip3ex

      {\bf Please}: 
      Put something here. 
        {\small\boldmath
\begin{align}
{\cal L} &= \prod_{i=1}^{n} 
\Pr(T_{wi} > t^0_{wi})^{(1-c_i)}
\Pr(T_{wi} = t_{wi} \cap T_{pi} > t^0_{pi} )^{c_i (1-d_i)}
\Pr(T_{wi} = t_{wi} \cap T_{pi} = t_{pi} )^{c_i d_i}. \nonumber
\end{align}
}

      \end{alertblock}

	\vskip2ex

      % -----------------------------------------------------------
      % 2-3
      % -----------------------------------------------------------
      \begin{block}{Another Sample Block}
      \vskip-2ex
      \begin{figure}[h]
	\hfill
	\begin{minipage}[t]{.49\textwidth}
	\begin{center}
	\caption{Figure Title Here}
		\includegraphics{figLeft}
	\end{center}
	\end{minipage}
	\hfill
	\begin{minipage}[t]{.49\textwidth}
	\begin{center}
	\caption{Figure Title Here}
		\includegraphics{figRight}
	\end{center}
	\end{minipage}
	\hfill
	\end{figure}
	\vskip3ex

R code ({\tt plotFigs.R}) accompanying this template will produce all the figures included in this poster (except for the Rice Logo).

\vskip1ex

{\bf Explain the figure here.}
	\begin{itemize}
	\item Put something here. Put something here. Put something here. Put something here. Put something here. 
	\item Put something here. Put something here. Put something here. Put something here. Put something here. 
	\end{itemize}

	\vskip3ex

{\bf Put something here.}
	\begin{itemize}
	\item Blur Blur Blur
	\item Yadi Yadi Yah
	\end{itemize}

      \end{block}

\end{column}

    % -----------------------------------------------------------
    % Start the third column
    % -----------------------------------------------------------
    \begin{column}{\onecolwid}
\vskip3ex

      % -----------------------------------------------------------
      % 3-0
      % -----------------------------------------------------------
      \begin{block}{Sample Block}
        \baselineskip=.7\baselineskip
\vskip2ex
        The following table is from the TeXShop template. 

	\vskip2ex

% Requires the booktabs if the memoir class is not being used
\begin{table}[htbp]
   \centering
   %\topcaption{Table captions are better up top} % requires the topcapt package
   \begin{tabular}{@{} lcr @{}} % Column formatting, @{} suppresses leading/trailing space
      \toprule
      \multicolumn{2}{c}{Item} \\
      \cmidrule(r){1-2} % Partial rule. (r) trims the line a little bit on the right; (l) & (lr) also possible
      Animal    & Description & Price (\$)\\
      \midrule
      Gnat      & per gram & 13.65 \\
                & each     &  0.01 \\
      Gnu       & stuffed  & 92.50 \\
      Emu       & stuffed  & 33.33 \\
      Armadillo & frozen   &  8.99 \\
      \bottomrule
   \end{tabular}
   \caption{Remember, \emph{never} use vertical lines in tables.}
   \label{tab:booktabs}
\end{table}
	
	\vskip2ex

	{\bf I honestly don't know what this table means.} 
	\begin{itemize}
	\item This is just
	\item an
	\item example.
	\end{itemize}

      \end{block}

\vskip3ex


      % -----------------------------------------------------------
      % 3-4
      % -----------------------------------------------------------
      \begin{alertblock}{Findings}
        \baselineskip=.7\baselineskip

      \vskip2ex

	\begin{figure}
	\begin{center}
	\caption{Effect of X on Y}
		\includegraphics{figResult}
	\end{center}
	\end{figure}
Explain the findings.
      \vskip2ex

	\end{alertblock}
      \vskip2ex

      % -----------------------------------------------------------
      % 3-4
      % -----------------------------------------------------------
      \begin{block}{Conclusion}
        \baselineskip=.7\baselineskip

      \vskip2ex

	{\bf Remember:}\\
	\begin{itemize} 
	\item You'd 
	\item better
	\item keep
	\item it
	\item simple!
	\end{itemize}

      \vskip2ex

	\end{block}
      \vskip2ex


	\end{column}
\end{columns}
\end{frame}
\end{document}

